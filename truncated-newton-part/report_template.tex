% This is a template for the final report, based on the guidelines in NO4LSP_Assignments_2526-1-5.pdf.
% You can use this as a guide to structure your final report.

\documentclass[11pt,a4paper]{article}

% ---------- Typography & layout ----------
\usepackage[T1]{fontenc}
\usepackage{lmodern}
\usepackage{microtype}
\usepackage[a4paper,margin=2.6cm]{geometry}
\usepackage{enumitem}
\usepackage{graphicx} % Required for inserting images
\usepackage{booktabs}
\usepackage{amsmath}
\usepackage{algorithm}    % floating algorithm
\usepackage{algpseudocode}% pseudocode
\usepackage{float}    
% table colors
\usepackage[table]{xcolor}
% Table grid
\arrayrulecolor{black!40} % 30% black (light gray)

% Nice headers/footers + links
\usepackage{fancyhdr}
\usepackage[hidelinks]{hyperref}

% Paragraph look
\setlength{\parindent}{0pt}
\setlength{\parskip}{0.55em}


% --- Bibliography Setup ---
\usepackage[
    backend=biber,      % The engine (best for Overleaf)
    style=numeric,      % Options: numeric, ieee, alphabetic, apa
    sorting=nyt         % Sort by Name, Year, Title
]{biblatex}

\addbibresource{references.bib} % Link your file



% Header/Footer setup
\pagestyle{fancy}
\fancyhf{}
\lhead{Numerical Optimization Report}
\rhead{A.A. 2025--2026}
\cfoot{\thepage}
\renewcommand{\headrulewidth}{0.4pt}

\title{\textbf{Numerical Optimization for Large Scale Problems Report}}
\author{% 
Matteo Aldovardi (student ID:\ 360426)\\ 
Andrea Nocchiero (student ID:\ 361794)
}


\begin{document}

\maketitle
\thispagestyle{empty}

\vspace{0.5em}
\hrule
\vspace{1.2em}

\section{Introduction}
% Brief introduction to the report (max. half a page).
% Explain the purpose of the report and what it contains.

\newpage

\section{Implemented Methods}
% Brief description of the implemented methods (max. one page per method).

\subsection{Modified Newton method with Back-tracking}
% This is your colleague's part.

\subsection{Truncated Newton Method}
% This is your part. You can This implementation follows the Inexact Newton framework \cite{nocedal2006numerical}. At each iteration $k$, the search direction $p_k$ is computed by approximately solving the Newton system $B_k p = -\nabla f(x_k)$ via Preconditioned Conjugate Gradient (PCG), truncated to handle non-convexity or achieve efficient convergence.

\subsection*{Algorithm Outline}
\begin{enumerate}
    \item \textbf{Forcing Term ($\eta_k$):} Determines the relative tolerance for the inner solver.
    \begin{itemize}
        \item $k=1$: $\eta_k = 0.5$.
        \item $k > 1$: \textbf{Superlinear:} $\min(0.5, \sqrt{\|\nabla f(x_{k-1})\|})$ or \textbf{Quadratic:} $\min(0.5, \|\nabla f(x_{k-1})\|)$.
    \end{itemize}

    \item \textbf{Preconditioning:} Supports $M = \text{diag}(B_k)$ or Modified Cholesky via \texttt{ichol} with diagonal shifting to ensure $M \succ 0$. The system $M y_j = r_j$ is solved at each inner step.

    \item \textbf{Inner Solver: Truncated PCG Updates:}
    The system $B_k p = -\nabla f(x_k)$ is solved starting with $z_0 = 0, r_0 = \nabla f(x_k)$. At each inner iteration $j$:
    
    
    
    \begin{itemize}
        \item \textbf{Curvature Check:} Compute $\kappa_j = d_j^T B_k d_j$.
        \begin{itemize}
            \item If $\kappa_j \le 0$ and $j=1$: Return $p_k = -\nabla f(x_k)$ (Steepest Descent).
            \item If $\kappa_j \le 0$ and $j > 1$: Truncate and return $p_k = z_j$.
        \end{itemize}
        \item \textbf{Iterative Updates:}
        \[ \alpha_j = \frac{r_j^T y_j}{\kappa_j}, \quad z_{j+1} = z_j + \alpha_j d_j, \quad r_{j+1} = r_j + \alpha_j B_k d_j \]
        \item \textbf{Conjugate Update:} With $y_{j+1} = M^{-1} r_{j+1}$ and $\beta_j = \frac{r_{j+1}^T y_{j+1}}{r_j^T y_j}$:
        \[ d_{j+1} = -y_{j+1} + \beta_j d_j \]
        \item \textbf{Truncation:} Stop when $\|r_j\| \le \eta_k \|\nabla f(x_k)\|$ or $j = \text{maxit}$.
    \end{itemize}

    \item \textbf{Descent Safeguard:} If $\nabla f(x_k)^T p_k \ge 0$ or $p_k$ contains NaNs, set $p_k = -\nabla f(x_k)$.

    \item \textbf{Line Search:} A backtracking approach finds $\alpha_k$ satisfying the Armijo condition:
    \[ f(x_k + \alpha_k p_k) \le f(x_k) + c_1 \alpha_k \nabla f(x_k)^T p_k \]
    with the usual break condition if line search fails.
    
    \item \textbf{Update:} $x_{k+1} = x_k + \alpha_k p_k$ until $\|\nabla f(x_k)\| < \text{tol}$.
\end{enumerate}

\subsection*{Remarks}
The code utilizes pre-allocation for the \texttt{history} struct to record function values, gradient norms, and CG iteration counts. Robustness is ensured via \texttt{condest(Bk)} monitoring and a fallback mechanism to steepest descent during line search failures or negative curvature detections. here.

\section{Test Problems}
% Introductory analysis of the test problems considered (max. one page per test problem).
% For each problem, report the mathematical formulation and the exact derivatives.
The objective function for the Broyden tridiagonal function, formulated as a nonlinear least-squares problem, is given by:
$$ F(x) = \frac{1}{2} \sum_{k=1}^{n} f_k(x)^2 $$
where the residual functions $f_k(x)$ are defined as:
$$ f_k(x) = (3 - 2x_k)x_k - x_{k-1} - 2x_{k+1} + 1 $$
This function is subject to the following boundary conditions:
$$ x_0 = 0 \quad \text{and} \quad x_{n+1} = 0 $$
The standard starting point for optimization algorithms is $x_i = -1$ for all $i=1, \dots, n$.

\subsubsection*{Gradient of the Objective Function}

The gradient of the objective function is required for derivative-based optimization methods. The factor of $\frac{1}{2}$ in the objective function simplifies the expression.
$$ \frac{\partial F}{\partial x_j} = \sum_{k=1}^{n} f_k(x) \frac{\partial f_k}{\partial x_j} $$
Using the Kronecker delta, $\delta_{ij}$, the partial derivative of the residual function $f_k(x)$ with respect to $x_j$ is:
$$ \frac{\partial f_k}{\partial x_j} = (3 - 4x_k)\delta_{kj} - \delta_{k,j+1} - 2\delta_{k,j-1} $$
Substituting this into the gradient expression and simplifying the sum by using the sifting property of the Kronecker delta, we arrive at a single compact formula. By defining $f_0 = 0$ and $f_{n+1} = 0$ to handle the boundaries, the gradient components are:
$$ \frac{\partial F}{\partial x_j} = f_j(x) (3 - 4x_j) - f_{j+1}(x) - 2 f_{j-1}(x) $$
This single expression is valid for all $j=1, \dots, n$.

\subsubsection*{Hessian of the Objective Function}

For second-order optimization methods, the Hessian of $F(x) = \frac{1}{2}\mathbf{f}(x)^T \mathbf{f}(x)$ is given by:
$$ \nabla^2 F(x) = J(x)^T J(x) + \sum_{k=1}^{n} f_k(x) \nabla^2 f_k(x) $$
where $J(x)$ is the Jacobian matrix of the vector of residuals $\mathbf{f}(x) = [f_1(x), \dots, f_n(x)]^T$.

The Jacobian matrix, $J(x)$, has a tridiagonal structure based on the partial derivatives of the residuals:
$$ J(x) = \begin{pmatrix}
3-4x_1 & -2 & 0 & \dots & 0 \\
-1 & 3-4x_2 & -2 & \dots & 0 \\
0 & -1 & 3-4x_3 & \dots & 0 \\
\vdots & \vdots & \ddots & \ddots & \vdots \\
0 & 0 & \dots & -1 & 3-4x_n
\end{pmatrix} $$
The second term involves the Hessians of the individual residual functions, $\nabla^2 f_k(x)$. The only non-zero second partial derivative of $f_k(x)$ is:
$$ \frac{\partial^2 f_k}{\partial x_k^2} = -4 $$
This means the summation term simplifies to a diagonal matrix:
$$ \sum_{k=1}^{n} f_k(x) \nabla^2 f_k(x) = \text{diag}(-4f_1(x), -4f_2(x), \dots, -4f_n(x)) $$
Combining the terms, the full Hessian of the objective function is:
$$ \nabla^2 F(x) = J(x)^T J(x) - 4 \cdot \text{diag}(f_1(x), f_2(x), \dots, f_n(x)) $$
The objective function for Problem 83, formulated as a nonlinear least-squares problem, is given by:
$$ F(x) = \frac{1}{2} \sum_{k=1}^{n} f_k(x)^2 $$
where the residual functions $f_k(x)$ are defined as:
$$ f_k(x) = 2x_k + h^2(x_k + \sin(x_k)) - x_{k-1} - x_{k+1} $$
with the constant $h = 1/(n+1)$. The function is subject to the following boundary conditions:
$$ x_0 = 0 \quad \text{and} \quad x_{n+1} = 1 $$
The standard starting point for optimization algorithms is $x_i = 1$ for all $i=1, \dots, n$.

\subsubsection*{Gradient of the Objective Function}

The gradient of the objective function is found using the chain rule, simplified by the $\frac{1}{2}$ factor in the objective function:
$$ \frac{\partial F}{\partial x_j} = \sum_{k=1}^{n} f_k(x) \frac{\partial f_k}{\partial x_j} $$
The partial derivative of the residual function $f_k(x)$ with respect to $x_j$ can be expressed compactly using the Kronecker delta, $\delta_{ij}$:
$$ \frac{\partial f_k}{\partial x_j} = (2 + h^2 + h^2\cos(x_k))\delta_{kj} - \delta_{k,j+1} - \delta_{k,j-1} $$
By defining $f_0 = 0$ and $f_{n+1} = 0$ to handle the boundaries, substituting this into the gradient expression and simplifying the sum yields a single general formula for the gradient components:
$$ \frac{\partial F}{\partial x_j} = f_j(x) (2 + h^2 + h^2\cos(x_j)) - f_{j+1}(x) - f_{j-1}(x) $$
This single expression is valid for all $j=1, \dots, n$.

\subsubsection*{Hessian of the Objective Function}

For second-order optimization methods, the Hessian of $F(x) = \frac{1}{2}\mathbf{f}(x)^T \mathbf{f}(x)$ is given by:
$$ \nabla^2 F(x) = J(x)^T J(x) + \sum_{k=1}^{n} f_k(x) \nabla^2 f_k(x) $$
where $J(x)$ is the Jacobian matrix of the vector of residuals $\mathbf{f}(x)$.

The Jacobian matrix, $J(x)$, is symmetric and tridiagonal:
$$ J(x) = \begin{pmatrix}
\alpha_1 & -1 & 0 & \dots \\
-1 & \alpha_2 & -1 & \dots \\
0 & -1 & \ddots & \dots \\
\vdots & \ddots & \ddots & -1 \\
0 & \dots & -1 & \alpha_n
\end{pmatrix} $$
where $\alpha_k = 2 + h^2 + h^2\cos(x_k)$.

The second term in the Hessian expression involves the Hessians of the individual residual functions, $\nabla^2 f_k(x)$. The only non-zero second partial derivative of $f_k(x)$ is:
$$ \frac{\partial^2 f_k}{\partial x_k^2} = -h^2\sin(x_k) $$
This simplifies the summation term to a diagonal matrix:
$$ \sum_{k=1}^{n} f_k(x) \nabla^2 f_k(x) = \text{diag}(-h^2 f_1(x)\sin(x_1), \dots, -h^2 f_n(x)\sin(x_n)) $$
Combining the terms, the full Hessian of the objective function is:
$$ \nabla^2 F(x) = J(x)^T J(x) - h^2 \cdot \text{diag}(f_1(x)\sin(x_1), \dots, f_n(x)\sin(x_n)) $$



\section{Finite Differences Implementation}
% If implemented, a brief description of the finite differences implementation (max. one page per test problem).
% Report the strategy you adopted for implementing the finite differences.

\section{Results}
% Tables and/or figures summarizing your results.

\subsection{Problem 31 - Modified Newton}
% Your colleague's results for problem 31.

\subsection{Problem 31 - Truncated Newton}
% Your results for problem 31.

\subsection{Problem 83 - Modified Newton}
% Your colleague's results for problem 83.

\subsection{Problem 83 - Truncated Newton}
% Your results for problem 83.

\section{Comments and Analysis}
% Comments and analyses on your results for the test problems (max. one page per test problem).
% Compare the results obtained, for example in terms of the number of iterations and computing time, commenting on your results also in view of the values of the parameters used and of the theory.

\section{Appendix: Code}
% Add the commented scripts/functions you implemented.
% Use sensible names for the variables and functions, and provide enough comments and explanations to make the code readable.

\subsection{Truncated Newton Code}
% \verbatiminput{path/to/your/code.m}


% Bibliography page
\newpage
\printbibliography[heading=bibintoc, title={References}]

\end{document}
